\documentclass[10pt]{article}
\usepackage[letterpaper]{geometry}
\geometry{verbose,tmargin=1in,bmargin=1in,lmargin=1in,rmargin=1in}
\usepackage{setspace}
\usepackage{ragged2e}
\usepackage{color}
\usepackage{titlesec}
\usepackage{graphicx}
\usepackage{float}
\usepackage{mathtools}
\usepackage{amsmath}
\usepackage[font=small,labelfont=bf,labelsep=period]{caption}
\usepackage[english]{babel}
\usepackage{indentfirst}
\usepackage{array}
\usepackage{makecell}
\usepackage[usenames,dvipsnames]{xcolor}
\usepackage{multirow}
\usepackage{tabularx}
\usepackage{arydshln}
\usepackage{caption}
\usepackage{subcaption}
\usepackage{xfrac}
\usepackage{etoolbox}
\usepackage{cite}
\usepackage{url}
\usepackage{dcolumn}
\usepackage{hyperref}
\usepackage{courier}
\usepackage{url}
\usepackage{esvect}
\usepackage{commath}
\usepackage{verbatim} % for block comments
\usepackage{enumitem}
\usepackage{hyperref} % for clickable table of contents
\usepackage{braket}
\usepackage{titlesec}
\usepackage{booktabs}
\usepackage{gensymb}
\usepackage{longtable}

% for circled numbers
\usepackage{tikz}
\newcommand*\circled[1]{\tikz[baseline=(char.base)]{
            \node[shape=circle,draw,inner sep=2pt] (char) {#1};}}


\titleclass{\subsubsubsection}{straight}[\subsection]

% define new command for triple sub sections
\newcounter{subsubsubsection}[subsubsection]
\renewcommand\thesubsubsubsection{\thesubsubsection.\arabic{subsubsubsection}}
\renewcommand\theparagraph{\thesubsubsubsection.\arabic{paragraph}} % optional; useful if paragraphs are to be numbered

\titleformat{\subsubsubsection}
  {\normalfont\normalsize\bfseries}{\thesubsubsubsection}{1em}{}
\titlespacing*{\subsubsubsection}
{0pt}{3.25ex plus 1ex minus .2ex}{1.5ex plus .2ex}

\makeatletter
\renewcommand\paragraph{\@startsection{paragraph}{5}{\z@}%
  {3.25ex \@plus1ex \@minus.2ex}%
  {-1em}%
  {\normalfont\normalsize\bfseries}}
\renewcommand\subparagraph{\@startsection{subparagraph}{6}{\parindent}%
  {3.25ex \@plus1ex \@minus .2ex}%
  {-1em}%
  {\normalfont\normalsize\bfseries}}
\def\toclevel@subsubsubsection{4}
\def\toclevel@paragraph{5}
\def\toclevel@paragraph{6}
\def\l@subsubsubsection{\@dottedtocline{4}{7em}{4em}}
\def\l@paragraph{\@dottedtocline{5}{10em}{5em}}
\def\l@subparagraph{\@dottedtocline{6}{14em}{6em}}
\makeatother

\newcommand{\volume}{\mathop{\ooalign{\hfil$V$\hfil\cr\kern0.08em--\hfil\cr}}\nolimits}

\setcounter{secnumdepth}{4}
\setcounter{tocdepth}{4}
\begin{document}

\title{ME 280a: HW 1}
\author{April Novak}

\maketitle

\section{Introduction and Objectives}

The purpose of this study is to solve a simple finite element (FE) problem and perform a convergence study to determine the number of elements needed to reach a specific tolerance. The Galerkin FE method is used, which for certain classes of problems possesses the ``best approximation property,'' a characteristic that signifies that the FE solution obtained is the best possible solution for a given mesh refinement and choice of shape functions. The mathematical procedure is described in Section \ref{sec:Procedure}.

\section{Procedure}
\label{sec:Procedure}

This section describes the mathematical process used to solve the following problem:

\begin{equation}
\label{eq:Problem}
\frac{d}{dx}\left(E(x)\frac{du}{dx}\right)=k^2\sin{\left(\frac{2\pi kx}{L}\right)}
\end{equation}

where \(E\) is the modulus of elasticity, \(u\) is the solution, \(k\) is a known constant, \(L\) is the problem domain length, and \(x\) is the spatial variable. The Galerkin FE method (FEM) achieves the best approximation property by expanding both \(u\) and a test function \(\psi\) in the same set of basis functions:

\begin{equation}
\label{eq:approx}
\begin{aligned}
u\approx u^N=\sum_{j=1}^{N}a_j\phi_j\\
\psi=\sum_{i=1}^{N}b_i\phi_i\\
\end{aligned}
\end{equation}

where \(u^N\) is the approximate solution. Galerkin's method is stated as:

\begin{equation}
r^N\cdot u^N=0
\end{equation}

where \(r^N\) is the residual. Hence, to formulate the weak form to Eq. \eqref{eq:Problem}, multiply Eq. \eqref{eq:Problem} through by \(\psi\) and integrate over all space, \(d\Omega\).

\begin{equation}
\int_{\Omega}^{}\frac{d}{dx}\left(E(x)\frac{du}{dx}\right)\psi d\Omega-\int_{\Omega}^{}k^2\sin{\left(\frac{2\pi kx}{L}\right)}\psi d\Omega=0
\end{equation}

Applying integration by parts to the first term:

\begin{equation}
-\int_{\Omega}^{}E(x)\frac{du}{dx}\frac{d\psi}{dx}d\Omega+\int_{\partial\Omega}^{}E(x)\frac{du}{dx}\psi d(\partial\Omega)-\int_{\Omega}^{}k^2\sin{\left(\frac{2\pi kx}{L}\right)}\psi d\Omega=0
\end{equation}

where \(\partial\Omega\) refers to one dimension lower than \(\Omega\), which for this case refers to evaluation at the endpoints of the domain. Hence, for this particular 1-D problem, the above reduces to:

\begin{equation}
\begin{aligned}
-\int_{0}^{L}E(x)\frac{du}{dx}\frac{d\psi}{dx}dx+ E(x)\frac{du}{dx}\psi\biggr\vert_{0}^{L}-\int_{0}^{L}k^2\sin{\left(\frac{2\pi kx}{L}\right)}\psi dx=0\\
\int_{0}^{L}E(x)\frac{du}{dx}\frac{d\psi}{dx}dx=\int_{0}^{L}k^2\sin{\left(\frac{2\pi kx}{L}\right)}\psi dx-E(x)\frac{du}{dx}\psi\biggr\vert_{0}^{L}\\
\end{aligned}
\end{equation}

Inserting the approximation in Eq. \eqref{eq:approx}:

\begin{equation}
\begin{aligned}
\int_{0}^{L}E(x)\frac{d\left(\sum_{j=1}^{N}a_j\phi_j\right)}{dx}\frac{d\left(\sum_{i=1}^{N}b_i\phi_i\right)}{dx}dx=\int_{0}^{L}k^2\sin{\left(\frac{2\pi kx}{L}\right)}\left(\sum_{i=1}^{N}b_i\phi_i\right)dx-E(x)\frac{d\left(\sum_{j=1}^{N}a_j\phi_j\right)}{dx}\left(\sum_{i=1}^{N}b_i\phi_i\right)\biggr\vert_{0}^{L}\\
\end{aligned}
\end{equation}

Recognizing that \(b_i\) appears in each term, the sum of the remaining terms must also equal zero (i.e. basically cancel \(b_i\) from each term).

\begin{equation}
\begin{aligned}
\int_{0}^{L}E(x)\frac{d\left(\sum_{j=1}^{N}a_j\phi_j\right)}{dx}\frac{d\phi_i}{dx}dx=\int_{0}^{L}k^2\sin{\left(\frac{2\pi kx}{L}\right)}\phi_idx-E(x)\frac{d\left(\sum_{j=1}^{N}a_j\phi_i\right)}{dx}\left(\phi_j\right)\biggr\vert_{0}^{L}\\
\end{aligned}
\end{equation}

This equation can be satisfied for each choice of \(j\), and hence can be reduced to:

\begin{equation}
\begin{aligned}
\int_{0}^{L}E(x)\frac{d\left(a_j\phi_j\right)}{dx}\frac{d\phi_i}{dx}dx=\int_{0}^{L}k^2\sin{\left(\frac{2\pi kx}{L}\right)}\phi_idx-E(x)\frac{d\left(a_j\phi_j\right)}{dx}\left(\phi_i\right)\biggr\vert_{0}^{L}\\
\end{aligned}
\end{equation}

There are two possible boundary conditions that can be imposed - essential boundary conditions, in which \(u\) is specified, and natural boundary conditions, where \(E(x)du/dx\) is specified. The Galerkin method specifies that the test functions \(\psi\) satisfy the homogeneous form of the essential boundary conditions. This produces a system of matrix equations of the form:

\begin{equation}
\textbf{K}\vv{x}=\vv{b}
\end{equation}

where:

\begin{equation}
\begin{aligned}
K_{ij}=\int_{0}^{L}E(x)\frac{d\phi_i}{dx}\frac{d\phi_j}{dx}dx\\
x_j=a_j\\
b_i=\int_{0}^{L}k^2\sin{\left(\frac{2\pi kx}{L}\right)}\phi_idx\\
\end{aligned}
\end{equation}

The above equation governs the entire domain. \(\textbf{K}\) is an \(n \times n\) matrix, where \(n\) is the number of global nodes. In order for these equations to be useful with Gaussian quadrature, they must be transformed to the master element which exists over \(-1\leq\xi\leq1\) with the Jacobian \(dx/d\xi\):

\begin{equation}
\begin{aligned}
K_{ij}=\int_{0}^{L}E(x)\frac{d\phi_i}{dx}\frac{d\phi_j}{dx}\frac{dx}{d\xi}d\xi\\
x_j=a_j\\
b_i=\int_{0}^{L}k^2\sin{\left(\frac{2\pi kx}{L}\right)}\phi_idx\\
\end{aligned}
\end{equation} 

For linear elements, the shape functions have the following form over the master element:

\begin{equation}
\begin{aligned}
N_1(\xi)=\frac{1-\xi}{2}\\
N_2(\xi)=\frac{1+\xi}{2}\\
\end{aligned}
\end{equation}

In order to verify that the program functions correctly, the solution is plotted with the analytical solution to Eq. \eqref{eq:Problem}. To determine the analytical solution, integrate Eq. \eqref{eq:Problem} once to obtain:

\begin{equation}
\label{eq:Problem2}
\frac{du}{dx}=-\frac{1}{E}k^2\cos{\left(\frac{2\pi kx}{L}\right)}\frac{L}{2\pi k}+C_1
\end{equation}

It has been assumed that \(E\) is not a function of \(x\). Integrating once more:

\begin{equation}
\label{eq:Problem2}
u(x)=-\frac{1}{E}k^2\sin{\left(\frac{2\pi kx}{L}\right)}\left(\frac{L}{2\pi k}\right)^2+C_1x
\end{equation}










\end{document}